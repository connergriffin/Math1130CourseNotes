\begin{example} Determining a line between two points.

    Find an equation of the line passing through \((-1,2)\) and \((3,7).\)
    \begin{solution}
        We start by finding the slope.

        \[m = \frac{7-2}{3-(-1)} = \frac{5}{4}.\]

        In point-slope form -- using \((-1,2)\) as the point -- the equation is
        \[y-2 = \frac{5}{4}(x-(-1)). \]

        We can simplify a little.

        \[y-2 = \frac{5}{4}(x+1)\]
    \end{solution}
\end{example}

Suppose the point on the line that we take is the \(y\)-intercept of the line. That is it is a point of the form \((0,b).\) Then in point slope form we have \(y-b = m(x-0)\) which can be rearranged into
\[y=mx+b.\]

%\begin{definition}
    The line with slope \(m\) and \(y\)-intercept \((0,b)\) is given by the equation
    \[y=mx+b.\]
    This is called the \emph{slope-intercept form} of the equation for the line.
\end{definition}

\begin{example}Rearranging the equations

Find the slope-intercept of the line given by \(y-1=2(x-3).\)

\begin{solution}
    Let's rearrange the equation into the form \(y=mx+b\) directly.
    \begin{gather*}
        y-1=2(x-3) \\
        y=2x-6+1 \\
        y=2x-5.
    \end{gather*}
    The \(y\)-intercept must be \((0,-5)\)

    Alternative solution:

    The slope is \(2.\) What is the \(y\)-intercept? We can find it by plugging \(0\) in for \(x.\)

    \begin{gather*}
        y-1 = 2(0-3) \\
        y=2(-3)+1 =-5.
    \end{gather*}

    Thus the equation for the line is \(y=3x-5.\)
\end{solution}
\end{example}

\begin{example}Equations for horizontal and vertical lines

The vertical line through \((2,3)\) is given by \(x=2.\)

The horizontal line through \((2,3)\) is \(y=3.\)

In general, the vertical line through \((a,b)\) is \(x=a\) and the horizontal line is \(y=b.\)
\end{example}