When graphing a quadratic, we can technically completely graph it from the vertex and \(y\)-intercept since these two points determine the quadratic. However, it will be helpful to determine a few other geometric details of the parabola. In addition to these two points we should determine if it opens upwards or opens downwards and it's \(x\)-intercepts.

\begin{question}
    Graph the quadratic function  \(f(x) = -\frac{1}{2}x^2+3x-\frac{5}{2}.\)
    \begin{solution}
        In this case, \(a = -\frac{1}{2},\) \(b = 3,\) \(c=-\frac{5}{2}.\)

        The \(y\)-intercept: \((0,-\frac{5}{2}).\)
        
        The vertex: \(\left(3,2\right).\) (Verify!)

        We know that our function opens downwards since \(a<0.\)

        Let's determine the \(x\)-intercepts.

        To do this we need to solve the equation \(f(x) = 0.\)
        Since \(f\) is quadratic we can use the quadratic formula.
        \begin{gather*}
            x = \frac{-3 \pm \sqrt{9-4(-\frac{1}{2})(-\frac{5}{2})}}{2(-\frac{1}{2})} \\
            x = 3 \pm \sqrt{9-5} \\
            x = 3\pm \sqrt{4} \\
            x= 3+2, 3-2 \\
            x=5,1
        \end{gather*}
        The \(x\)-intercepts are \((5,0)\) and \((1,0).\)

        To graph the parabola graph each of the points and then draw a parabola meeting all of them which opens downwards.

        \desmos{cwojiwqyrv}{100}{600}
    \end{solution}
\end{question}