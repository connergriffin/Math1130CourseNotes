\begin{example}
    Suppose \(f\) is a linear function, \(f(2)=6\) and \(f(3)=2.\) Find an expression for \(f(x).\)
    \begin{solution}
        In the case of two points \((x_0,y_0)\) and \((x_1,y_1)\) we made sure that the indices lined up in our formula for the slope. Now that we are dealing with two points in the graph of a function we may treat the input value as we did the index. That is if \((x,f(x))\) and \((y,f(y))\) are two points in the graph of a linear function then the formula for the slope is \(\frac{f(x)-f(y)}{x-y}.\) In order to get this right we need the \(x\)'s and \(y\)'s to line up.

        \[m = \frac{f(3)-f{2}}{3-2} = \frac{2-6}{3-2} = -4.\]

        Equation in point-slope form: \(y-f(2) = -4(x-2).\)
        Solving for \(y:\) \(y = -4x+8+6 = -4x+14.\)

        Thus \(f(x) = -4x+14.\)
    \end{solution}
\end{example}