\begin{definition}
    A function \(f\) is \emph{one-to-one} is equivalent to the statement if \(f(x) = f(y)\) then \(x=y.\) 
\end{definition}

All linear functions are one-to-one. Indeed,
\begin{gather*}
    mx+b = my+b \\
    mx=my \\
    x=y.
\end{gather*}

The function \(f(x)=x^2\) -- with domain the entire real line -- is not one-to-one since \(f(-1)=f(1)\) but \(-1\ne 1.\) If we instead take the domain to be only the positive real numbers, then this function is one-to-one. One can think of this as the ``horizontal line test.''

\begin{theorem}[Horizontal Line Test]
    A function is one-to-one if every horizontal line passes through the graph of the function at most one time.
\end{theorem}

In this section, the important thing about one-to-one functions is that if we have an equation \(y = f(x)\) where \(f\) is some one-to-one function, then we can practically consider either \(x\) or \(y\) as the independent variable. In the supply-demand problems we will see, we can treat either supply or demand as the independent variable. Usually, we will consider price \((p)\) as the independent variable and supply \((q)\) as the dependent variable. Sometimes it will make sense to swap this relationship.