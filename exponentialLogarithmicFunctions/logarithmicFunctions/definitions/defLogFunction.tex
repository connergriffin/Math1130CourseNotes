The \emph{inverse} of the exponential function with base \(b,\) \(f\left(x\right) = b^x,\) is the logarithmic function with base \(b.\) This means that we are ``undoing'' the exponential and hence answering the question ``Given a positive real number, \(x\), what value goes in the box?'' for the expression \(b^{\Box} = x.\) Note that all exponential functions are one-to-one on their entire domain and hence invertible on their entire domain.

\begin{definition}
    For any \(b\) in \(\left(0,\infty\right)-\{1\}\), the \emph{logarithm with base \(b\)} is the function \(\log_b:(0,\infty) \to (-\infty,\infty)\) defined by \(y=\log_b\left(x\right)\) if and only if \(b^y = x.\)
\end{definition}

Being an inverse we have the following identities.

\begin{example}
    For all real numbers \(x,\) \(\log_b\left(b^x\right) =x\) and for all positive real numbers \(x,\) \(b^{\log_b\left(x\right)} =x.\)
\end{example}

Remember that the exponential with base \(e\) is our most important exponential function; likewise the logarithm with base \(e\) is the most important logarithmic function. It is called the \emph{natural logarithm} and in this class it will be denoted \(\ln\left(x\right)\) instead of \(\log_e\left(x\right).\) The other important logarithmic function is the logarithm with base \(10.\) We will write \(\log\) for \(\log_{10}\) in this class.

\desmos{hhydtagmkj}{100}{600}