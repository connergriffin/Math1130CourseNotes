\begin{theorem}
    The following rules of logarithms can be derived from the corresponding rule of exponents. Suppose \(b\) belongs to \(\left(0,\infty\right)-\{1\}\) and \(x,y\) are positive real numbers.
    \begin{enumerate}
        \item \(\log_b\left(xy\right) = \log_b\left(x\right)+\log_b\left(y\right)\)
        \item \(\log_b\left(x^y\right) = y\log_b\left(x\right)\)
        \item \(\log_b\left(\frac{x}{y}\right) = \log_b\left(x\right) - \log_b\left(y\right)\) (Note that this is a combination of a and b)
        \item \(\log_b\left(x\right)\log_x\left(b\right) = 1\) (Note that this follows from b)
        \item \(\log_b\left(1\right) = 0\)
        \item \(\log_b\left(\frac{1}{x}\right) = -\log_b\left(x\right)\)
    \end{enumerate}
\end{theorem}

The following are the corresponding rules of exponents:
\begin{enumerate}
    \item[e(a)] \(b^{x+y} = b^xb^y\)
    \item[e(b)] \(a^{xy} = \left(a^x\right)^y = \left(a^y\right)^x\)
\end{enumerate}

That is \(e(a)\) corresponds to \((a)\) and \(e(b)\) corresponds to \((b).\) The rest of the log rules presented follow from \((a)\) and \((b).\)