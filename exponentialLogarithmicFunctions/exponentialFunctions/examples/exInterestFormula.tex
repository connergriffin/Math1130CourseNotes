Exponential functions are important in business as we use them to calculate compounding interest.

The compounding interest formula has three variables, \(P,r,n.\)

The variable \(P\) represents the \emph{principal} which is the initial investment that will be accruing interest. The variable \(r\) is the rate that will be applied to the principal and \(n\) is the number of \emph{years}\footnote{This can be any unit of time. It is most often taken to be years, months or 3 months (that is quarterly). The unit that is used is called the compound frequency.} that the principal has been at rate \(r.\)

The total sum, \(S,\) after \(n\) at rate \(r\) being applied to the principal \(P\) is
\[S = P(1+r)^n.\]

\begin{example}
    Suppose \(\$100\) is invested for \(10\) years at \(3\%\) compounded annually. Find the compound amount. (i.e. the total sum after \(10\) years at \(3\%\)) Find the compound interest.
    \begin{solution}
        \[S = 100(1+0.03)^{10} \approx 134.39\]

        Thus the total compound interest earned is \(S-P = 34.39\)
    \end{solution}
\end{example}

\begin{example}
    Suppose \(\$100\) is invested for \(10\) years at \(3\%\) compounded quarterly. Find the compound amount. (i.e. the total sum after \(10\) years at \(3\%\)) Find the compound interest.
    \begin{solution}
        In \(10\) years time, how many units of \(3\) months will occur? Well, \(10\) years is \(120\) months. Thus we have that the interest will compound \(\frac{120}{3} = 40\) times! That is the \(n\) that we need to take for our interest formula.

        \[S = 100(1+0.03)^{40} \approx 326.20\]

        Thus the total compound interest earned is \(S-P = 226.20\)
    \end{solution}
\end{example}