\documentclass{ximera}

\title{Section 0.8 Quadratic Equations}
\author{Conner Griffin}

\begin{document}
\begin{abstract}
    In this section we will cover several methods for finding solutions to quadratic equations. We will see examples of higher degree polynomial equations may easily be factored.
\end{abstract}
\maketitle

\begin{definition}
    A \emph{quadratic equation} in the variable \(x\) is an equation that can be written in the form
    \[ax^2+bx+c=0\]
    where \(a,b,c\) are real numbers and \(a\ne 0.\)
\end{definition}

\begin{question}
    Which of the following are quadratic equations
    \begin{selectAll}
        \choice[correct]{\(x^2+x+1=0\)}
        \choice[correct]{\(w^2=2\)}
        \choice{\(y^2+3y = y^2 +7\)}
    \end{selectAll}
\end{question}

Option \(c\) is not correct since when in the form \(ay^2+by+c=0\) the coefficient of \(y^2\) is \(0.\)

We have two methods for finding all solutions to a quadratic equation.

\begin{enumerate}
    \item[Method 1] Factoring
    
        This method is quick if the solutions are integers! (\(\dots,-2,-1,0,1,2,\dots\))
    \item[Method 2] Quadratic formula
    
        Always works. There is a formula to memorize and often there will be some simplification to do.
\end{enumerate}
\newpage
\underline{Factoring Examples} %Method 1
\begin{question}
Solve for \(x:\) \(x^2+6x+8=0.\)
\begin{expandable}{}{}
\begin{solution}
    Consider that \(\left(x+a\right)\left(x+b\right) = x^2+\left(a+b\right)x +ab.\)
    Our goal is to find \(a\) and \(b.\)
    We should first list the ways we can factor \(8\) as a product of a pair of integers.
    \[8 = 8\cdot 1 \quad \textrm{or} \quad \left(-8\right)\left(-1\right) \quad \textrm{or} \quad 4\cdot 2 \quad \textrm{or} \quad \left(-4\right)\left(-2\right)\]
    We then sum these factors. Whichever pair sums to \(6\) will be our \(a\) and \(b.\)
    \[8+1 =9 \quad -8-1=-9 \quad 4+2=6 \checkmark.\]
    Thus the left hand side (LHS) factors as \( \left(x+4\right)\left(x+2\right) \) and our equation can also be written
    \[\left(x+4\right)\left(x+2\right)  = 0.\]
    Recall that if \(a\cdot b =0\) then \(a =0\) or \(b=0.\)
    In our case this means that
    \[x+4=0 \quad \textrm{or} \quad x+2=0.\]
    Thus \(-4\) and \(-2\) are the solutions to our quadratic equation.

    The set of solutions to our quadratic equation is \(\{-4,-2\}.\)
\end{solution}
\end{expandable}
\end{question}

\begin{question}
Solve for \(x:\) \(\left(y-2\right)\left(y+3\right) =6.\)

\textbf{Tip:} Note that this is not already appropriately factored. In the previous question, we relied on the fact that if \(ab=0\) then \(a=0\) or \(b=0.\) The same is not true for \(ab=6.\) We must first rewrite the equation in the form \(x^2+bx+c=0.\) It should be your habit to always start these problems by rearranging the equation into this ``standard form''.
\begin{expandable}{}{}
    \begin{solution}
        \begin{gather*}
            \left(y-2\right)\left(y-3\right)=6 \\
            y^2 -2y+3y -6 =6 \\
            y^2 -y -12 =0
        \end{gather*}
        Now that it is in the standard form, we apply the factoring technique from above to get
        \[\left(y+3\right)\left(y-4\right)=0.\]

        The set of solutions to our quadratic equation is \(\{-3,4\}.\)
    \end{solution}
\end{expandable}
\end{question}

\begin{question}
    Solve for \(t:\) \(4t^3+t^2 -4t-1 =0.\)
\begin{expandable}{}{}
    \begin{solution}
        This is \emph{not} a quadratic equation since there is a \(t^3\) term. However, this is a special kind of higher degree polynomial equation that is simple to factor.

        \begin{gather*}
            4t^3+t^2 -4t-1 =0 \\
            t^2\left(4t+1\right)-\left(4t+1\right) =0\\
            \left(t^2-1\right)\left(4t+1\right)=0
        \end{gather*}

        Hence, \(4t+1 =0\) -- in which case, \(t=-\frac{1}{4}\) -- or \(t^2-1=0.\) Polynomials of the form \(t^2-a^2\) can always be factored as \(\left(t-a\right)\left(t+a\right).\) This is called the ``difference of squares formula.'' Remember it! Thus from the quadratic equation we have
        \begin{gather*}
            t^2-1=0 \\
            \left(t+1\right)\left(t-1\right) =0.
        \end{gather*}
        Therefore, \(-1\) and \(1\) are solutions.
        
        The complete set of solutions to our equation is \(\{-1,-\frac{1}{4},1\}.\)
    \end{solution}
\end{expandable}
\end{question}

\underline{Quadratic Formula Examples} %Method 2

For the general quadratic equation, \(ax^2+bx+c=0,\) the two solutions are
\[x = \frac{-b\pm \sqrt{b^2-4ac}}{2a}.\]

\begin{question}
    Solve for \(x:\) \(2x^2+\frac{3}{2}x-3=0.\)
    \begin{expandable}{}{}
        \begin{solution}
            \begin{flalign*}
                x&=\frac{-3/2 \pm \sqrt{9/4-4\left(2\right)\left(-3\right)}}{2\left(2\right)} \\
                &= \frac{-3/2 \pm \sqrt{9/4+24}}{4} \\
                &= \frac{2}{2}\frac{-3/2 \pm \sqrt{9/4 +24}}{4} \\
                &= \frac{-3 \pm \sqrt{9+24\left(4\right)}}{8} \\
                &= \frac{-3 \pm \sqrt{105}}{8}
            \end{flalign*}
            The complete set of solutions to our equation is \(\{\frac{-3+\sqrt{105}}{8},\frac{-3-\sqrt{105}}{8}\}.\)
        \end{solution}
    \end{expandable}
\end{question}

\begin{question}
    Solve for \(z:\) \(2z^2+z+3=0.\)
    \begin{expandable}{}{}
        \begin{solution}
            \begin{flalign*}
                z&=\frac{-1 \pm \sqrt{1-4\left(2\right)\left(3\right)}}{2\left(2\right)} \\
                &= \frac{-1 \pm \sqrt{-23}}{4}
            \end{flalign*}
            Since \(\sqrt{-23}\) is not a real number, ther are no solutions to this equation!

            The set of real solutions is \(\emptyset.\)

            This is the empty set. The set which has no elements.
        \end{solution}
    \end{expandable}
\end{question}

\underline{Quadratic Form Equations}

\begin{definition}
    A \emph{quadratic form equation} in the variable \(x\) is an equation that can be written in the form
    \[ax^{2k}+bx^k+c=0\]
    where \(a,b,c,k\) are real numbers and \(a\ne 0\) and \(k\ne 0.\)
\end{definition}

\begin{question}
    Solve for \(x:\) \(\frac{1}{x^6}+\frac{2}{x^3}+1=0.\)

    \textbf{Tip:} This is a quadratic form equation since it has the form \(\left(x^{-3}\right)^2+2x^{-3}+1=0.\) That is, in this case, \(k=-3.\) When dealing with a quadratic form equation you should start by making the substitution \(u=x^k.\)
    \begin{expandable}{}{}
        \begin{solution}
            Let \(u=\frac{1}{x^3}\).
            \begin{gather*}
                \frac{1}{x^6}+\frac{2}{x^3}+1=0 \\
                u^2+2u+1=0 \\
                \left(u+1\right)\left(u+1\right)=0.
            \end{gather*}
            Hence, \(u=-1.\) Since \(u = \frac{1}{x^3}\), we have \(-x^3=1\) and then \(x=-1.\)

            The complete set of solutions to this equation is \(\{-1\}.\)
        \end{solution}
    \end{expandable}
\end{question}

\begin{question}
    Solve for \(x:\) \(\frac{1}{2}x+2\sqrt{x}-3=0.\)
    \begin{expandable}{}{}
        \begin{solution}
            This is another example of a quadratic form equation.
            Let \(u=\sqrt{x}\).
            \begin{gather*}
                \frac{1}{2}x+2\sqrt{x}-3=0\\
                \frac{1}{2}u^2+2u-3=0
            \end{gather*}
            From the quadratic formula we see that \(u =  -2+\sqrt{10}, \ -2-\sqrt{10}.\)

            Thus we have \(\sqrt{x} = -2+\sqrt{10}\) and \(\sqrt{x} = -2-\sqrt{10}\)

            Squaring each side we get \(x = 14-4\sqrt{10}\) and \(x = 14+4\sqrt{10}\)

            The complete set of solutions to this equation is \(\{14-4\sqrt{10},14+4\sqrt{10}\}.\)
        \end{solution}
    \end{expandable}
\end{question}

\end{document}
