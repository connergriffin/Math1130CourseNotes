\begin{question}
    Solve for \(x:\) \(\frac{1}{x^6}+\frac{2}{x^3}+1=0.\)

    \textbf{Tip:} This is a quadratic form equation since it has the form \(\left(x^{-3}\right)^2+2x^{-3}+1=0.\) That is, in this case, \(k=-3.\) When dealing with a quadratic form equation you should start by making the substitution \(u=x^k.\)
    \begin{expandable}{}{}
        \begin{solution}
            Let \(u=\frac{1}{x^3}\).
            \begin{gather*}
                \frac{1}{x^6}+\frac{2}{x^3}+1=0 \\
                u^2+2u+1=0 \\
                \left(u+1\right)\left(u+1\right)=0.
            \end{gather*}
            Hence, \(u=-1.\) Since \(u = \frac{1}{x^3}\), we have \(-x^3=1\) and then \(x=-1.\)

            The complete set of solutions to this equation is \(\{-1\}.\)
        \end{solution}
    \end{expandable}
\end{question}

\begin{question}
    Solve for \(x:\) \(\frac{1}{2}x+2\sqrt{x}-3=0.\)
    \begin{expandable}{}{}
        \begin{solution}
            First note that any solution to this equation must be positive or \(0\) since we cannot evaluate the left hand side at negative numbers.
            This is another example of a quadratic form equation.
            Let \(u=\sqrt{x}\).
            \begin{gather*}
                \frac{1}{2}x+2\sqrt{x}-3=0\\
                \frac{1}{2}u^2+2u-3=0
            \end{gather*}
            From the quadratic formula we see that \(u =  -2+\sqrt{10}, \ -2-\sqrt{10}.\)

            Thus we have \(\sqrt{x} = -2+\sqrt{10}\) and \(\sqrt{x} = -2-\sqrt{10}\)

            Squaring each side we get \(x = 14-4\sqrt{10}\) and \(x = 14+4\sqrt{10}\)

            Note that both answers are positive and hence each is a solution to our equation.
            
            The complete set of solutions to this equation is \(\{14-4\sqrt{10},14+4\sqrt{10}\}.\)
        \end{solution}
    \end{expandable}
\end{question}