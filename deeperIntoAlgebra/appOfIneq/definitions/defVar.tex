We begin this section by reviewing a few business terms that will be relevant to our applied problems. In a manufacturing setting, \emph{fixed cost} (FC) is the cost of everything which \emph{is not} dependent on the level of production. (e.g. rent) \emph{Variable cost} (VC) is the cost of everything which \emph{is} dependent on the level of production. (e.g. material) \emph{Cost} (C) is the sum of variable cost and fixed cost.
\[C = FC +VC.\]
\emph{Revenue} (R) is the money received for selling the manufactured product. It is quantity (Q) times price (P)
\[R = QP.\]
\emph{Profit} is revenue minus cost.
\[P = R-C.\]