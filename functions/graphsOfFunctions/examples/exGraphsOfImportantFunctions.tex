\begin{example}
    Graph of the square root function.

    Let's first make a table of function values that we can compute by hand.

    \begin{table}
        \centering
        \begin{tabular}{|c|c|}
            \hline
            \(x\) & \(\sqrt{x}\)\\[1.5ex] 
            \hline
            \(0\) & \(0\) \\[1.5ex] 
            \hline
            \(\frac{1}{4}\) & \(\frac{1}{2}\) \\[1.5ex] 
            \hline
            \(\frac{4}{9}\) & \(\frac{2}{3}\) \\[1.5ex] 
            \hline
            \(1\) & \(1\) \\[1.5ex] 
            \hline
            \(\frac{9}{4}\) & \(\frac{3}{2}\) \\[1.5ex]
            \hline \(4\) & \(2\) \\
            \hline
        \end{tabular}
        \caption{A table of function values for the square root function}
    \end{table}

    Let's plot these points and sketch the curve through them.

    \desmos{yosbaaxrvv}{100}{600}
\end{example}

\begin{example}
    Graph of the absolute value function.

    Let's first make a table of function values that we can compute by hand.

    \begin{table}
        \centering
        \begin{tabular}{|c|c|}
            \hline
            \(x\) & \(\left|x\right|\)\\[1.5ex] 
            \hline
            \(-1\) & \(1\) \\[1.5ex] 
            \hline
            \(-\frac{1}{2}\) & \(\frac{1}{2}\) \\[1.5ex] 
            \hline
            \(0\) & \(0\) \\[1.5ex] 
            \hline
            \(1\) & \(1\) \\[1.5ex]
            \hline
        \end{tabular}
        \caption{A table of function values for the absolute value function}
    \end{table}

    Let's plot these points and sketch the curve through them.

    \desmos{fwqctpwrac}{100}{600}
\end{example}

\begin{example}
    Graph of the case-defined function \[f(x) = \begin{cases}
        x & \textrm{if} \ 0 \le x < 2 \\
        x/2 -1 & \textrm{if} \ 2 \le x \le 4
    \end{cases}\]

    \begin{table}
        \centering
        \begin{tabular}{|c|c|}
            \hline
            \(x\) & \(f(x)\)\\[1.5ex] 
            \hline
            \(0\) & \(0\) \\[1.5ex] 
            \hline
            \(1\) & \(1\) \\[1.5ex] 
            \hline
            \(2\) & \(0\) \\[1.5ex] 
            \hline
            \(4\) & \(1\) \\[1.5ex]
            \hline
        \end{tabular}
        \caption{A table of function values for the absolute value function}
    \end{table}

    \desmos{f8ztlw2tly}{100}{600}

    Note that we do an empty circle to represent what \(f(2)\) \emph{would} be \emph{if} \(f(x) = x\) when \(x=2.\) This is \emph{not} the case because we have a \(<2\) instead of a \(\le2\) for the first rule.
\end{example}