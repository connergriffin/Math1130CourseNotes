The \(y\)-intercept of a function is always \((0,f(0))\) and is often times very easy to compute. On the other hand, we can precisely locate \(x\)-intercepts algebraically in very specific circumstances. The \(x\)-intercepts are the points where \((x,f(x)) = (x, 0).\) Thus we need to solve the equation \(f(x) = 0.\) The \(x\) values which solve this equation are called the \emph{roots} of \(f.\) As we have seen in previous sections, we can easily solve this equation when \(f\) is a linear or quadratic polynomial function. We have seen examples of \emph{quadratic form functions}, some higher degree polynomial functions, some square root functions, and some rational functions for which we can solve the equation \(f(x)=0.\) For many functions there will be no method of precisely determining the roots of the function and we have to \emph{approximate} them using a computer. There are actually formula's like the the quadratic formula for 3rd and 4th degree polynomial equations! (They are very complicated. See \href{https://en.wikipedia.org/wiki/Cubic_equation#General_cubic_formula}{the cubic equation} and \href{https://en.wikipedia.org/wiki/Quartic_function#/media/File:Quartic_Formula.svg}{the quartic equation}.) There is no formula for 5th and higher degree equations.

\begin{question}
    Find the \(x\)- and \(y\)- intercepts of the function \(f\left(x\right) = \frac{2x}{3}-1.\) For this problem, you should adjust the sliders on the embedded Desmos graph above so that the graph represents this function.
    \begin{solution}
        Let's start with the \(y\)-intercepts.

        \[f(0) = \frac{0}{3}-1=-1.\]

        The \(y\)-intercept is \((0,-1).\)

        The \(x\)-intercepts:

        \begin{gather*}
            \frac{2x}{3}-1=0 \\
            \frac{2x}{3} = 1 \\
            x= \frac{3}{2}
        \end{gather*}

        There is only one \(x\) intercept and it is \((\frac{3}{2},0).\)
    \end{solution}
\end{question}

\begin{question}
    Determine the \(t\)- and \(y\)-intercepts of the function \(g\left(t\right) = \frac{2}{t}-1.\)
    \begin{solution}
        Since the variable is \(t\), we call the horizontal axis the \(t\)-axis and refer to the intercepts there as the \(t\)-intercepts.
        
        The \(y\)-intercept:

        Note that \(0\) is \emph{not} in the domain of \(f\) and hence there is \emph{no} \(y\)-intercept.

        The \(t\)-intercepts:

        \begin{gather*}
            \frac{2}{t}-1 = 0 \\
            \frac{2}{t} = 1 \\
            2 = t
        \end{gather*}

        The \(t\)-intercept is \((2,0).\)
    \end{solution}
\end{question}