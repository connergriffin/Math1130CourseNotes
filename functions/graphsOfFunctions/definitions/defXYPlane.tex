As was mentioned in Section 2.1, formally functions from the real line to the real line are subsets of the cartesian product \(\left(-\infty,\infty\right) \times \left(-\infty,\infty\right)\) which satisfy specific conditions. We can visualize the cartesian product as the \(xy\)-coordinate plane. Generally, the point of intersection for the \(x\)- and \(y\)-axes is taken to be the point \(\left(0,0\right)\) which we call the origin. If we would like to \emph{plot} an ordered pair say, \(\left(6,3.5\right),\) then a mark is made \(6\) units horizontally and \(3.5\) units vertically.

\begin{image}
\begin{tikzpicture}
    \draw[<->] (0,-0.2) --(0,4.2);
    \draw[<->] (-0.2, 0) -- (6.2, 0);

    \foreach \x in {1,2,3,4}{
        \draw (-0.1,\x) -- (0.1, \x);
        \node[left] at (-0.1,\x) {$\x$};
    }
    \foreach \x in {1, 2, 3, 4,5,6} {
        \draw (\x, -0.1) -- (\x, 0.1);
        \node[below] at (\x, -0.1) {$\x$};
    }
    \draw[blue, very thick] (0,0) -- (0,3.5);
    \draw[orange!60, very thick] (0,0) -- (6,0);
    \draw[dashed,blue, very thick] (6,0) -- (6,3.5);
    \draw[dashed,orange!60, very thick] (0,3.5) -- (6,3.5);
    \filldraw[black] (6,3.5) circle (2pt) node[anchor=west]{$(6,3.5)$};
    \end{tikzpicture}
    %\caption{The point \(\left(6,3.5\right)\) plotted on the \(xy\)-plane.}
\end{image}
