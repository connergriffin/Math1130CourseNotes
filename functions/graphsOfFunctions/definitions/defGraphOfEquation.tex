\begin{definition}
    An \emph{equation in the variables \(x\) and \(y\)} is a statement that two expressions involving only the variables \(x\) and \(y\) are equivalent.
\end{definition}

\begin{question} Which of the following are equations in two variables?
    \begin{selectAll}
        \choice[correct]{\(x^2+y^2 = 1\)}
        \choice[correct]{\(\frac{1}{2}= \frac{1}{x^2+1}\)}
        \choice{\(x^2+y^2\)}
        \choice[correct]{\(x^3-xy+y^3=0\)}
    \end{selectAll}
\end{question}

\begin{definition}
    A solution to an equation in the variables \(x\) and \(y\) is an ordered pair \((a,b)\) such that when \(a\) is plugged in for \(x\) and \(b\) for \(y\) the equality is true.
\end{definition}

\begin{definition}
    The \emph{graph of an equation} is an approximate plot of all of it's solutions in the \(xy\)-plane.
\end{definition}

There is an important equation in two variables that we derive from functions. If \(f(x)\) is a function, the set of solutions to the equation \(y=f(x)\) is all of the points of the form \((x,f(x)).\) Thus the graph of this equation is the same as the graph of the function. It is \emph{not} the case that every equation in two variables can be written in the form \(y=f(x)\) for some function in \(x!\)

Some graphs of important equations:

\desmos{yjghqr7015}{100}{600}

\desmos{yceoozxoyk}{100}{600}

\desmos{1ikvx8xix8}{100}{600}

\desmos{ea5uknfdiv}{100}{600}

Notice that none of these pass the vertical line test and thus cannot represent functions in a the variable \(x!\)