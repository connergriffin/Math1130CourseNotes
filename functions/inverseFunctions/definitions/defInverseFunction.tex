Related to the identity function is the inverse of a function \(f.\)

\begin{definition}
    The inverse of a function \(f\) from the real line to the real line is the unique function \(f^{-1}\) such that
    \[\left(f \circ f^{-1}\right)\left(x\right)=I\left(x\right)\]
    and
    \[\left(f^{-1}\circ f\right)\left(x\right) - I\left(x\right).\]
\end{definition}

In some sense, the inverse of \(f\) ``undoes'' \(f.\) For example, if \(f(x) = 2x-7\) we might ask, what goes in the box here \(2\Box-7=15.\) We can solve this using algebra. When we do so, we are applying \emph{inverse} functions to determine the value that goes in the box. 