\begin{question}
    Suppose consumers will buy \(q\) units of a product at a price of \(\frac{300}{q}+15\) dollars per unit. How much will the price per unit need to be to sell \(100\) units? How much revenue is made from selling \(100\) units at this price? Express revenue as a function of \(q.\)
    \begin{solution}
        Price is given as a function of quantity. That is \(p(q) = \frac{300}{q}+15.\) Then \(p(100) = \frac{300}{100}+15 = 18.\)

        \(Revenue = 18\cdot100 = 1800.\)

        Revenue as a function of quantity: \(R(q) = q \left(\frac{300}{q}+15\right) = 300+15q.\)
    \end{solution}
\end{question}

Recall from section 1.3 that we worked a problem where we needed the revenue function above to be greater than \(\$ 9450.\) Consider the Desmos graph of the revenue function below.

\desmos{fcaltpejxv}{100}{600}