Suppose we have the two functions \(f\left(x\right)=\sqrt{x}\) and \(g\left(x\right) = x^2+2.\) For any real number \(x,\) \(x^2+2\) is a positive real number and hence we can apply \(f\) to it. That is we can define a new function \(h\left(x\right) = f\left(x^2+2\right) = \sqrt{x^2+2}.\) Since we have evaluated \(f\) at \(g\) we typically write \(f(g(x))\) for the expression and we label \(h = f \circ g.\) That is
\[\left(f \circ g\right) \left(x\right) = f\left(g(x)\right) = \sqrt{x^2+2}.\]

\begin{definition}
    For functions \(g:X \to Y\) and \(f:Y \to Z\) The composition of \(f\) with \(g\) is the function \(f\circ g: X \to Z\) defined by
    \[f\circ g (x) = f(g(x))\]
    where the domain of \(f \circ g\) are all of the real numbers \(x\) such that \(x\) is in the domain of \(g\) and \(g(x)\) is in the domain of \(f.\)
\end{definition}