Suppose we have the two functions \(f\left(x\right)=\sqrt{x}\) and \(g\left(x\right) = \sqrt{x^2+2}.\) For any real number \(x,\) both \(\sqrt{x}\) and \(\sqrt{x^2+2}\) are real numbers. We can add real numbers. So, for any real number \(x,\) \(\sqrt{x} + \sqrt{x^2+2}\) is also a real number. We can define a new function \(h(x) = \sqrt{x} + \sqrt{x^2+2}.\) Since we have added \(f\) and \(g\) we label \(h\) as \(f+g.\) That is 
\[\left(f+g\right)\left(x\right) = \sqrt{x}+\sqrt{x^2+2}.\]

We can do this in general for any one of \(+,-,\cdot, /.\)

\begin{definition}
    Let \(f,g\) be functions from \(X\) -- a subset of the real line -- to \(\left(-\infty,\infty\right).\) Define the following operations on functions:
    \begin{itemize}
        \item \(\left(f+g\right)\left(x\right) = f\left(x\right) + g\left(x\right)\)
        \item \(\left(f-g\right)\left(x\right) = f\left(x\right) - g\left(x\right)\)
        \item \(\left(fg\right)\left(x\right) = f\left(x\right)g\left(x\right)\)
        \item \(\left(\frac{f}{g}\right)\left(x\right) = \frac{f\left(x\right)}{g\left(x\right)}\) for \(g\left(x\right)\ne 0.\)
    \end{itemize}
\end{definition}

The domain of each of these new functions is all real numbers which belong to the domain of \(f\) \emph{and} the domain of \(g.\)

\begin{example}
    Find the domain of \(f+g\) given that \(f\left(x\right) = \sqrt{1-x}\) and \(g\left(\right) = \sqrt{x+1}.\)
    \begin{solution}
        The domain of \(f\) is all \(x\) such that \(1-x \ge 0.\) That is \(1 \ge x.\) The domain of \(g\) is \(x+1 \ge 0.\) That is \(x \ge -1.\) The domain of \(f+g\) then is all \(x\) such that \(1 \ge x \ge -1.\) In interval notation it is \(\left[-1,1\right].\)
    \end{solution}
\end{example}