Recall our compound interest formula.

\begin{definition}
    For an original principal \(P\), the formula
    \[S = P\left(1+r\right)^n\]
    gives the \emph{compound amount} \(S\) at the end of \(n\) interest periods at the periodic rate of \(r.\) 
\end{definition}

%review example

\begin{example}
    Suppose that \(\$1000\) is invested at a nominal rate of \(7.5\%\) compounded quarterly. How much will be in the account after \(5\) years? How much interest is earned?
    \begin{solution}
        Remember, we need to convert the annual rate to a quarterly rate. To do this we divide by the number of times that interest is applied in a year. Since it is compounding \emph{quarterly}, this number is \(4.\)

        Quarterly rate: \(\frac{0.075}{4} = 0.01875\)

        We need the number of times that interest compounds in \(5\) years.

        Interest periods in \(5\) years: \(4\cdot 5=20.\)

        Compound amount: \(S = 1000\left(1+0.01875\right)^{20} = 1000\left(1.01875\right)^{20} \approx 1449.95\)

        Interest earned: \(1449.95 -1000 = 449.95\)
    \end{solution}
\end{example}