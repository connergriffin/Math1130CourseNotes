We might like to answer the question ``How much do I need to invest to have \(S\) amount after \(n\) years at \(r\) rate?'' We can answer this question by rearranging the compound interest formula.

\[S = P\left(1+r\right)^n \rightarrow P=S\left(1+r\right)^{-n}.\]

Now we have the principal as a function of the compound amount! In this context we usually say \(P\) is the ``present value'' and \(S\) is the ``future value.''

\begin{definition}
    The principal \(P\) that must be invested at the periodic rate of \(r\) for \(n\) interest periods so that the compound amount is \(S\) is given by
    \[P = S\left(1+r\right)^{-n}\]
    and is called the \emph{present value} of \(S.\)
\end{definition}